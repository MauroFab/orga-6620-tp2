\documentclass[11pt,a4paper, spanish]{article}
\usepackage[T1]{fontenc}
\usepackage[utf8]{inputenc}
\usepackage[spanish]{babel}
\selectlanguage{spanish}
\usepackage{graphicx}
\usepackage{listings}

\lstset{
    language=C,
    tabsize=4,
    basicstyle=\fontsize{11}{13}\ttfamily\footnotesize,
    showspaces=false,
    showstringspaces=false,
    captionpos=b,
    breaklines=true,
    literate={á}{{\'a}}1
        {ã}{{\~a}}1
        {é}{{\'e}}1
        {ó}{{\'o}}1
        {í}{{\'i}}1
        {ñ}{{\~n}}1
        {¡}{{!`}}1
        {¿}{{?`}}1
        {ú}{{\'u}}1
        {Í}{{\'I}}1
    {Ó}{{\'O}}1
}

\usepackage{multirow}
\usepackage{float}
\usepackage[caption = false]{subfig}

\setcounter{secnumdepth}{0}

\begin{document}

%  FRONTPAGE

\begin{titlepage}
  \noindent%
  \begin{tabular}{@{}p{\textwidth}@{}}
    \vspace{0.2cm}
    \begin{center}
    \Huge{\textbf{
      Memorias caché
    }}
    \end{center}
    \begin{center}
      \Large{
         66:20 Organizaci\'on de Computadoras
      }
    \end{center}
    \vspace{0.2cm}\\
  \end{tabular}
  \vspace{4 cm}
  \begin{center}
    {\large
      Trabajo práctico 2
    }\\
    \vspace{0.6cm}
    {\Large
      Axel Lijdens (95772)\\
      Eduardo R Madariaga (90824)\\
      Mauro Toscano (96890)
    }
  \end{center}
  \vfill
  \begin{center}
  Univesidad de Buenos Aires - FIUBA
  \end{center}
\end{titlepage}


\tableofcontents
\pagebreak
% Prologo

\section{Objetivos}

Familiarizarse con el funcionamiento de la memoria caché implementando una simulación de una caché dada.

\section{Alcance}

Este trabajo práctico es de elaboración grupal, evaluación individual, y de carácter obligatorio para todos alumnos del curso.

\section{Requisitos}

El trabajo deberá ser entregado personalmente, en la fecha estipulada, con una carátula que contenga los datos completos de todos los integrantes.

Además, es necesario que el trabajo práctico incluya (entre otras cosas, ver sección 8), la presentación de los resultados obtenidos, explicando, cuando corresponda, con fundamentos reales, las causas o razones de cada resultado obtenido. Por este motivo, el día de la entrega deben concurrir todos los integrantes del grupo.

El informe deber\'a respetar el modelo de referencia que se encuentra en el grupo, y se valorar\'an aquellos escritos usando la herramienta TEX / LATEX.

\section{Recursos}

Este trabajo práctico debe ser implementado en C, y correr al menos en Linux.

\section{Fecha de entrega}

La última fecha de entrega y presentación ser\'a el jueves 17 de mayo de 2018.

\section{Introducción}\label{informe}
La memoria a simular es una caché [1] asociativa por conjuntos de dos
vías, de 1KB de capacidad, bloques de 32 bytes, política de reemplazo LRU
y política de escritura WT/$\sim$ WA. Se asume que el espacio de direcciones es
de 12 bits, y hay entonces una memoria principal a simular con un tamaño
de 4KB. Estas memorias pueden ser implementadas como variables globales.
Cada bloque de la memoria caché deberá contar con su metadata, incluyendo
el bit V y el tag.

% Desarrollo

\section{Implementaci\'on del programa}

Para el dise\~no del programa se empez\'o por establecer los par\'ametros de entrada necesarios:

\begin{itemize}
\item Un archivo de entrada
\end{itemize}

Teniendo en cuenta lo propuesto se escribi\'o el siguiente mensaje de ayuda:

\begin{lstlisting}[numbers=left, tabsize=2, basicstyle=\fontsize{11}{13}\ttfamily, frame=single, caption={Mensaje de ayuda del programa}]
Uso: ./tp [archivo]
\end{lstlisting}

Asimismo, se debieron tener en cuenta y validar los siguientes casos en donde alg\'un factor es incorrecto y el mensaje devuelto por el probrama debe ser explicativo del error:

\begin{itemize}
\item \textbf{No se pudo abrir algun archivo:} Este error indica que el archivo indicado por parámetro de entrada no pudo ser abierto.
\item \textbf{Error de formato:} El archivo de entrada no respetaba el formato especificado en el enunciado.
\end{itemize}

Una vez que el archivo se abrió exitosamente, se procede a parsear y ejecutar las acciones en forma secuencial.
El programa imprime un detalle de los accesos a la caché simulada con el siguiente formato:

\begin{itemize}
\item \textit{WM:} Indica un Write-Miss en la caché.
\item \textit{WH[n][m][o]:} Indica un Write-Hit en la vía \textit{n}, bloque \textit{m} y offset \textit{o}.
\item \textit{RM[n][m]:} Indica un Read-Miss y una consecuente carga del bloque \textit{m} en la vía \textit{n}.
\item \textit{RH[n][m][o]:} Indica un Read-Hit en la caché en la vía \textit{n}, bloque \textit{m} y offset \textit{o}.
\item \textit{MR:} Corresponde al Miss rate calculado.
\end{itemize}

Además, las operaciones de lectura indican el valor leído (desde la caché en caso de un RH o desde memoria en caso de un RM).

Como solamente hay 2 vías, fue posible implementar el reemplazo LRU utilizando solamente 1 bit, el cuál indica
el estado "nuevo" si está en 1 y el estado "viejo" si está en cero. De esta forma no resultó necesario implementar
un algoritmo complejo para llevar la cuenta de uso de cada bloque.

Debido a que los bloques son de 32 bytes, se requieren 5 bits para el offset. La caché
tiene un tamaño de 1024 bytes, cada vía tiene 512 bytes. Los bloques son de 32 bytes
y por lo tanto hay 16 grupos, por lo que se necesitan 4 bits para el índice.

Para direccionar la memoria principal se requieren 12 bits en total, por lo tanto los
3 bits restantes se utilizan para el tag.

\begin{table}[h!]
  \begin{center}
    \label{tab:table1}
    \begin{tabular}{|c|c|c|}
        \cline{1-3}
        tag (3 bits) & index (4 bits) & offset (5 bits)\\
        \cline{1-3}
    \end{tabular}
    \caption{Dirección de memoria}
  \end{center}
\end{table}

\begin{table}[h!]
  \begin{center}
    \label{tab:table1}
    \begin{tabular}{|c|c|c|c|}
        \cline{1-4}
        tag (3 bits) & V (1 bit) & LRU (1 bit) & datos (32 bytes)\\
        \cline{1-4}
    \end{tabular}
    \caption{Bloque en la caché}
  \end{center}
\end{table}

\section{Compilación del programa}

Para compilar el programa basta ejecutar la siguiente línea:

\begin{center}
\textbf{make}
\end{center}

\section{Corridas de prueba}

A continuación se muestran los resultados de las ejecuciones del programa para cada archivo de pruebas:

\begin{lstlisting}[numbers=left, tabsize=2, basicstyle=\fontsize{11}{13}\ttfamily, frame=single, caption={prueba1.mem}]
W 0, 16
R 0
R 1024
R 8
R 2050
R 3074
W 8, 12
R 8
R 8
W 3072, 255
W 2048, 10
R 0
MR
\end{lstlisting}

\begin{lstlisting}[numbers=left, tabsize=2, basicstyle=\fontsize{11}{13}\ttfamily, frame=single, caption={Resultado prueba1.mem}]
WM
RM[0][0]    -> 16
RM[1][0]    -> 0
RH[0][0][8] -> 0
RM[1][0]    -> 0
RM[0][0]    -> 0
WM
RM[1][0]    -> 12
RH[1][0][8] -> 12
WH[0][0][0]=255
WM
RH[1][0][0] -> 16
MR=67
\end{lstlisting}

\begin{lstlisting}[numbers=left, tabsize=2, basicstyle=\fontsize{11}{13}\ttfamily, frame=single, caption={prueba2.mem}]
R 0
R 31
W 32, 10
R 32
W 32, 20
R 32
R 1040
R 2064
R 32
R 32
MR
\end{lstlisting}

\begin{lstlisting}[numbers=left, tabsize=2, basicstyle=\fontsize{11}{13}\ttfamily, frame=single, caption={Resultado prueba2.mem}]
RM[0][0]    -> 0
RH[0][0][31] -> 0
WM
RM[0][1]    -> 10
WH[0][1][0]=20
RH[0][1][0] -> 20
RM[1][0]    -> 0
RM[0][0]    -> 0
RH[0][1][0] -> 20
RH[0][1][0] -> 20
MR=50
\end{lstlisting}

\begin{lstlisting}[numbers=left, tabsize=2, basicstyle=\fontsize{11}{13}\ttfamily, frame=single, caption={prueba3.mem}]
W 0, 1
W 1, 2
W 2, 3
W 3, 4
W 4, 5
R 0
R 1
R 2
R 3
R 4
MR
\end{lstlisting}

\begin{lstlisting}[numbers=left, tabsize=2, basicstyle=\fontsize{11}{13}\ttfamily, frame=single, caption={Resultado prueba3.mem}]
WM
WM
WM
WM
WM
RM[0][0]    -> 1
RH[0][0][1] -> 2
RH[0][0][2] -> 3
RH[0][0][3] -> 4
RH[0][0][4] -> 5
MR=60
\end{lstlisting}

\begin{lstlisting}[numbers=left, tabsize=2, basicstyle=\fontsize{11}{13}\ttfamily, frame=single, caption={prueba4.mem}]
W 0, 1
W 1, 2
W 2, 3
W 3, 4
W 4, 5
R 0
R 1
R 2
R 3
R 4
R 1024
R 2048
R 0
R 1
R 2
R 3
R 4
MR
\end{lstlisting}

\begin{lstlisting}[numbers=left, tabsize=2, basicstyle=\fontsize{11}{13}\ttfamily, frame=single, caption={Resultado prueba4.mem}]
WM
WM
WM
WM
WM
RM[0][0]    -> 1
RH[0][0][1] -> 2
RH[0][0][2] -> 3
RH[0][0][3] -> 4
RH[0][0][4] -> 5
RM[1][0]    -> 0
RM[0][0]    -> 0
RM[1][0]    -> 1
RH[1][0][1] -> 2
RH[1][0][2] -> 3
RH[1][0][3] -> 4
RH[1][0][4] -> 5
MR=53
\end{lstlisting}

\begin{lstlisting}[numbers=left, tabsize=2, basicstyle=\fontsize{11}{13}\ttfamily, frame=single, caption={prueba5.mem}]
R 0
R 1024
R 3072
R 2048
R 0
R 3072
MR
\end{lstlisting}

\begin{lstlisting}[numbers=left, tabsize=2, basicstyle=\fontsize{11}{13}\ttfamily, frame=single, caption={Resultado prueba5.mem}]
RM[0][0]    -> 0
RM[1][0]    -> 0
RM[0][0]    -> 0
RM[1][0]    -> 0
RM[0][0]    -> 0
RM[1][0]    -> 0
MR=100
\end{lstlisting}

\section{Validación de resultados}

Para validar los resultados, haremos el trabajo de la cache manualmente, y contrastaremos con los resultados obtenidos.

\subsection{Primer Test}

W 0, 16\\ 
Miss compulsivo. No se trae el bloque a cache. Misses: 1\\

R 0\\ 
Miss compulsivo. Se trae el bloque al conjunto 0. Misses: 2\\

R 1024\\
Miss compulsivo. Se trae el bloque al conjunto 0. Misses: 3\\

R 8 \\
Hit\\

R 2050\\
Miss compulsivo. Se trae el bloque al conjunto 0, se quita el que inicia con la posición 1024. Misses: 4\\

R 3074 \\
Miss compulsivo. Se trae el bloque al conjunto 0, se quita el que inicia con la posición 0. Misses:4\\


W 8, 12 \\
Miss de conflicto. Misses:5\\

R 8 \\
Miss de conflicto. Se trae el bloque al conjunto 0, se quita el que inicia con la posición 2048. Misses:6\\

R 8 \\
Hit\\
 
W 3072, 255\\
Hit\\

W 2048, 10 \\
Miss de conflicto. Se trae el bloque del 2048. Se saca el de 0. Misses :7\\

R 0 \\
Miss de conflicto Misses:9\\

Tenemos entonces 9 misses en 12 operaciones. Esto es un miss rate de 2/3, en porcentaje aproximadamente \%67. Este resultado es identico al simulado en el programa\\

\subsection{Segundo Test}

R 0 \\
Miss compulsivo. Se trae el bloque al conjunto 0. Misses: 1\\

R 31\\
Hit\\

W 32, 10\\
Miss. Se trae el bloque al conjunto 1. Misses: 2\\

R 32\\
Hit\\

W 32, 20\\
Hit\\

R 32\\
Hit\\

R 1040 7\\
Miss compulsivo. Este bloque comienza en la posición 1024. Se lo el bloque al conjunto 0.  Misses: 3\\

R 2064 8\\ 
Miss compulsivo. Este bloque comienza en la posición 2048. Se lo trae va al conjunto 0. Se saca el bloque que comienza con la dirección 0. Misses: 4\\

R 32\\
Miss de conflicto. Se trae el bloque al conjunto 0. Se saca el que comienza en la 1024.
Misses: 5\\

R 32\\
Hit\\

Tenemos entonces 5 misses en 10 operaciones. Esto es un miss rate de un \%50. Este resultado es identico al simulado en el programa.\\

\subsection{Tercer Test}

W 0, 1
W 1, 2
W 2, 3
W 3, 4
W 4, 5\\
Misses: 5\\

R 0\\
Miss compulsivo. Misses: 6\\

R 1
R 2
R 3
R 4\\
4 Hits\\

Tenemos entonces 6 misses en 10 operaciones. Esto es un miss rate de un \%60. Este resultado es identico al simulado en el programa.\\

\subsection{Cuarto Test}
W 0, 1
W 1, 2
W 2, 3
W 3, 4
W 4, 5\\
Misses: 5\\
R 0 \\
Misses: 6

R 1
R 2
R 3
R 4\\
4 hits

R 1024
R 2048\\ 
Misses: 8.Se saca al bloque que comienza con la posición de la memoria 0 de la cache\\ 

R 0
\\ Misses: 9
R 1
R 2
R 3
R 4
\\ 4 hits

Tenemos entonces 9 misses en 17 operaciones. Esto es un miss rate de un \%53. Este resultado no es identico al simulado en el programa.\\

\subsection{Quinto Test}
R 0
R 1024
R 3072
R 2048
R 0
R 3072\\

Y esto son todos misses, tenemos dos vias, todo va al conjunto 0, y tengo mas de 2 accesos a memoria antes de volver a buscar un dato. El miss rate es del \%100 y es idéntico al simulado.

\section{Conclusiones}

La implementación de la simulación nos premitió estudiar mas en detalle el funcionamiento de una caché de 2 vías. Las cachés
del estilo WT/~WA tienen la ventaja de ser simples de implementar, ya que no hace falta considerar el estado "dirty",
porque los datos siempre se escriben a la memoria principal. La desventaja es que no hay mejora de velocidad en las
escrituras (se comportan de la misma forma que si no existiera la caché).


\subsection{Problemas encontrados a lo largo del proyecto}

Es dificil llevar la cuenta de los bloques actualizados/reemplazados por las lecturas, por lo que desarrollar
una salida detallada fue de gran utilidad para detectar errores de programación.


% SOlO codigos fuente hacia abajo, no hace falta editar!

\newpage

\section{C\'odigos fuente}

Repositorio: https://github.com/MauroFab/orga-6620-tp2

\subsection{MakeFile}

\begin{lstlisting}[numbers=left, tabsize=2, basicstyle=\fontsize{11}{13}\ttfamily, frame=single, caption={makefile}]

tp: main.c cache.c cache.h
	@gcc -ggdb -std=c99 -Wall -Wpedantic -Werror main.c cache.c -o tp -lm
	@echo "LD tp"

clean:
	@rm -f tp

.PHONY: clean

\end{lstlisting}

\subsection{Main}

\begin{lstlisting}[numbers=left, tabsize=2, basicstyle=\fontsize{11}{13}\ttfamily, frame=single, caption={makefile}]
#include <stdbool.h>
#include <stdio.h>
#include <stdlib.h>
#include <string.h>
#include "cache.h"

#define MEM_SIZE 4096  // 4 KiB

char memory[MEM_SIZE];

/*
 * Reinicia la tasa miss e invalida la cache
 */
void init() {
  cache_init();
}

/*
 * Si el adress esta en la cache devuelve su valor
 * Si no lo carga de memory y devuelve -1
 */
static char read_byte(int address) {
  if (address >= MEM_SIZE) {
    return -1;
  }
  char rv = cache_read_byte(address, memory);
  if (rv == -1) {
    return memory[address];
  }
  return rv;
}

static int write_byte(int address, unsigned char value) {
  if (address >= MEM_SIZE) {
    return -1;
  }
  return cache_write_byte(address, value, memory);
}

static bool _process_file(FILE *file) {
  while (!feof(file)) {
    unsigned int address;
    unsigned int value;
    char command = '\0';

    if (fscanf(file, "%c", &command) != 1) {
      /* verifica que se trate el fin de archivo */
      return (feof(file) > 0);
    }

    switch (command) {
      case 'R':
        if (fscanf(file, "%u", &address) != 1) {
          return false;
        }
        printf("%d\n", read_byte(address));
        break;

      case 'W':
        if (fscanf(file, "%u, %u", &address, &value) != 2) {
          return false;
        }
        write_byte(address, value);
        break;

      case 'M':
        if (fscanf(file, "%c", &command) != 1) {
          return false;
        }
        if (command != 'R') {
          return false;
        }

        printf("MR=%d\n", cache_get_miss_rate());
        break;

      default:
        return false;
    }

    /* lee el fin de linea */
    if (fread(&value, 1, 1, file) == 0) {
      break;
    }
  }

  return true;
}

int main(int argc, const char *argv[]) {
  if (argc != 2) {
    printf("Uso: %s [archivo]\n", argv[0]);
    return EXIT_FAILURE;
  }

  const char *filename = argv[1];
  FILE *file = fopen(filename, "r");
  if (file == NULL) {
    printf("No se pudo abrir el archivo.\n");
    return EXIT_FAILURE;
  }

  init();

  if (!_process_file(file)) {
    goto error;
  }

  fclose(file);
  return EXIT_SUCCESS;

error:
  fclose(file);
  printf("Error de formato.\n");
  return EXIT_FAILURE;
}

\end{lstlisting}

\begin{lstlisting}[numbers=left, tabsize=2, basicstyle=\fontsize{11}{13}\ttfamily, frame=single, caption={makefile}]

#ifndef CACHE_H_
#define CACHE_H_

#include <stdint.h>

/**
 * bits: 31            11    9       5        0
 *       --------------------------------------
 *       |             | tag | index | offset |
 *       --------------------------------------
 */

/** Tamaño total de la cache */
#define CACHE_SIZE 1024
/** Tamaño de cada bloque de la cache */
#define CACHE_BS 32
/** Número de vías. */
#define CACHE_WAYS_NUM 2

/** Tamaño de cada vía. */
#define CACHE_WAY_SIZE (CACHE_SIZE / CACHE_WAYS_NUM)
#define CACHE_BLOCKS_PER_WAY ((CACHE_SIZE / CACHE_WAYS_NUM) / CACHE_BS)

void cache_init();
char cache_read_byte(int address, const char *memory);
int cache_write_byte(int address, unsigned char value, char *memory);
int cache_get_miss_rate();

#endif


\end{lstlisting}

\begin{lstlisting}[numbers=left, tabsize=2, basicstyle=\fontsize{11}{13}\ttfamily, frame=single, caption={makefile}]

#include "cache.h"
#include <math.h>
#include <stdint.h>
#include <string.h>

#define ENABLE_LOG 1

#if ENABLE_LOG
#include <stdio.h>
#define LOG(...) (printf(__VA_ARGS__))
#else
#define LOG(...) \
  do {           \
  } while (0)
#endif

#define BLOCK(way, index) (cache.entries[way][index])

/** Marca la entrada del cache como válida. */
#define SET_VALID(way, index) (BLOCK(way, index).is_valid = 1)
/** Devuelve "true" si el cache está marcado como válido. */
#define IS_VALID(way, index) (BLOCK(way, index).is_valid == 1)

/** Marca la entrada del cache como nueva (LRU). */
#define SET_NEW(way, index) (BLOCK(way, index).is_new = 1)
/** Marca la entrada del cache como vieja (LRU). */
#define SET_OLD(way, index) (BLOCK(way, index).is_new = 0)
/** Retorns "true" si la entrada es vieja. */
#define IS_OLD(way, index) (BLOCK(way, index).is_new == 0)

/** Obtiene el offset desde la dirección de memoria. */
#define OFFSET(address) (address & 0x1F)
/** Obtiene el index desde la dirección de memoria. */
#define INDEX(address) ((address >> 5) & 0x0F)
/** Obtiene el tag desde la dirección de memoria. */
#define TAG(address) ((address >> 9) & 0x07)

/** Bloque de datos en el cache. */
typedef uint8_t cache_block_t[CACHE_BS];

/** Entrada en el cache. */
typedef struct {
  uint8_t tag : 3;
  uint8_t is_valid : 1;
  uint8_t is_new : 1;
  cache_block_t data;
} cache_entry_t;

/** Cache. */
typedef struct {
  cache_entry_t entries[CACHE_WAYS_NUM][CACHE_BLOCKS_PER_WAY];
  double miss_count;
  double hit_count;
} cache_t;

/** Instancia global del cache. */
static cache_t cache;

/**
 * @brief Carga un bloque de memoria principal a a la cache.
 *
 * @param index El indice del conjunto de la cache.
 * @param tag El tag del bloque.
 * @param mem Memoria principal.
 * @return La vía en la cual se cargó el bloque.
 */
static void _load_block(int index, int tag, const char *mem) {
  int way = 0;
  if (IS_OLD(0, index)) {
    way = 0;
  } else {
    way = 1;
  }

  cache_entry_t *entry = &cache.entries[way][index];

  /* copia los datos y actualiza los metadatos */
  memcpy(entry->data, mem, CACHE_BS);

  SET_NEW(way, index);
  SET_OLD(!way, index);
  SET_VALID(way, index);
  entry->tag = tag;

  LOG("RM[%d][%d]    -> ", way, index);
}

/**
 * @brief Lee un byte del cache y actualiza los metadatos.
 *
 * @param way Via a leer.
 * @param index Indice del groupo.
 * @param offset Offset del bloque.
 * @return char Dato.
 */
static char _read_cache(int way, int index, int offset) {
  cache.entries[way][index].is_new = 1;
  cache.entries[!way][index].is_new = 0;
  char value = cache.entries[way][index].data[offset];

  LOG("RH[%d][%d][%d] -> ", way, index, offset);
  return value;
}

/**
 * @brief Inicializa el cache.
 *
 */
void cache_init() {
  memset(&cache, 0, sizeof(cache_t));
}

/**
 * @brief Lee un byte del cache.
 *
 * @param address Dirección del byte a leer.
 * @param memoria Memoria.
 * @return unsigned char Byte leído.
 */
char cache_read_byte(int address, const char *memory) {
  int index = INDEX(address);
  int offset = OFFSET(address);

  if (IS_VALID(0, index) && TAG(address) == cache.entries[0][index].tag) {
    cache.hit_count++;
    return _read_cache(0, index, offset);
  } else if (IS_VALID(1, index) && TAG(address) == cache.entries[1][index].tag) {
    cache.hit_count++;
    return _read_cache(1, index, offset);
  } else {
    cache.miss_count++;
    _load_block(index, TAG(address), memory);
    return -1;
  }
}

/**
 * @brief Escribe en la memoria y si en necesario ne el cache.
 *
 * @param address Dirección de memoria en la que se escribe.
 * @param value Valor a escribir.
 * @param memory Memoria principal.
 * @return int 0 si fue un hit, -1 si fue un miss.
 */
int cache_write_byte(int address, unsigned char value, char *memory) {
  int index = INDEX(address);
  int offset = OFFSET(address);

  int rv = 0;

  if (IS_VALID(0, index) && TAG(address) == cache.entries[0][index].tag) {
    cache.hit_count++;
    *(cache.entries[0][index].data + offset) = value;
    cache.entries[0][index].is_new = 1;
    cache.entries[1][index].is_new = 0;
    LOG("WH[%d][%d][%d]=%d\n", 0, index, offset, value);
  } else if (IS_VALID(1, index) && TAG(address) == cache.entries[1][index].tag) {
    cache.hit_count++;
    *(cache.entries[1][index].data + offset) = value;
    cache.entries[1][index].is_new = 1;
    cache.entries[0][index].is_new = 0;
    LOG("WH[%d][%d][%d]=%d\n", 1, index, offset, value);
  } else {
    cache.miss_count++;
    LOG("WM\n");
    rv = -1;
  }

  /* WT: siempre escribe a memoria */
  *(memory + address) = value;
  return rv;
}

/**
 * @brief Devuelve el miss rate redondeado al entero mas cercano.
 *
 * @return uint64_t miss rate.
 */
int cache_get_miss_rate() {
  return round((cache.miss_count * 100) / (cache.miss_count + cache.hit_count));
}

\end{lstlisting}


\end{document}
\grid
